\documentclass[conference]{IEEEtran}
%\documentclass[sigconf]{acmart}
\makeatletter
\def\ps@headings{%
\def\@oddhead{\mbox{}\scriptsize\rightmark \hfil \thepage}%
\def\@evenhead{\scriptsize\thepage \hfil \leftmark\mbox{}}%
\def\@oddfoot{}%
\def\@evenfoot{}}
\makeatother
\pagestyle{empty}
\usepackage{url}
\usepackage{graphicx,subfigure}
\usepackage{epstopdf}
\usepackage{amsmath}
\usepackage{algorithm}
\usepackage{algpseudocode}
\usepackage{amsmath}
\usepackage{amssymb}
\usepackage{amsthm}
\usepackage{epsfig}
\newtheorem{theorem}{Theorem}
\renewcommand{\algorithmicrequire}{\textbf{Input:}} % Use Input in the format of Algorithm
\renewcommand{\algorithmicensure}{\textbf{Output:}} % Use Output in the format of Algorithm
\usepackage{amsfonts}
%\newtheorem{theorem}{Theorem}[section]
\newtheorem{mydef}{Definition}[section]
%\newtheorem{lemma}{Lemma}[section]
\usepackage{multirow}
\usepackage{color}
\usepackage{array}
\usepackage{listings}
\usepackage{hyperref}
\usepackage[underline=true]{pgf-umlsd}
\newcommand{\tabincell}[2]
{\begin{tabular}
		{@{}#1@{}}#2\end{tabular}}
\usepackage{setspace}
\renewcommand{\labelitemi}{$\vcenter{\hbox{\tiny$\bullet$}}$}


\hyphenation{op-tical net-works semi-conduc-tor}




\begin{document}



\title{Group 4 Final Project}

\author{\IEEEauthorblockN{1\textsuperscript{st} Riley Hawley}
\IEEEauthorblockA{\textit{Dept. of Computer Science} \\
\textit{Stevens Institute of Technology}\\
Hoboken, United States \\
rhawley2@stevens.edu}
\and
\IEEEauthorblockN{2\textsuperscript{nd} Eddie Kuang}
\IEEEauthorblockA{\textit{Dept. of Computer Science} \\
\textit{Stevens Institute of Technology}\\
Hoboken, United States \\
ekuang@stevens.edu}
\and
\IEEEauthorblockN{3\textsuperscript{rd} Jince Shi}
\IEEEauthorblockA{\textit{Dept. of Computer Science} \\
\textit{Stevens Institute of Technology}\\
Hoboken, United States \\
jshi39@stevens.edu}
\and
\IEEEauthorblockN{4\textsuperscript{rd} Eric Tashji}
\IEEEauthorblockA{\textit{Dept. of Computer Science} \\
\textit{Stevens Institute of Technology}\\
Hoboken, United States \\
etashji@stevens.edu}
}

\maketitle


\begin{abstract}
The problem statement for the project is: "Republican voters tend to reside more in rural areas, whereas Democrat voters tend to 
reside more in urban areas. Republicans want to increase its turnout in urban areas by 5\% and Democrats want to increase its 
turnout in rural areas by 5\%." The Machine Learning algorithms we are using to solve the problem are the K-means, Logistic Regression,
Linear Regression, and Decision Tree algorithms. Thus far, our experimental results indicate that while rural voters tend to vote 
Republican and urban voters tend to vote Democrat, a big factor that plays into that result is the variance in the individuals' 
most important issues. 
\end{abstract}

\section{Introduction}
% The section includes SEVERAL paragraphs summarizing your project. It is like the extended version of Abstract - you may use one paragraph for each of these parts - problem statement, dataset description, machine learning algorithms you will use to solve the problem, experimental results, and how your solutions are better as compared to existing solutions. Please try to limit Introduction to one page.
The problem statement for the project is: "Republican voters tend to reside more in rural areas, whereas Democrat voters tend to 
reside more in urban areas. Republicans want to increase its turnout in urban areas by 5\% and Democrats want to increase its 
turnout in rural areas by 5\%."

There will be two data sets: one for voters in rural areas and one for voters in urban areas. The data will include each individual’s
age, gender, marital status, number of children, occupation, salary, standard of living, party membership, rate of voter turnout, 
most important issue (when polled), engagement with Democrat or Republican ground campaign, and percentage of votes for Democrat or 
Republican candidates for the past 8 years.

We will use four machine-learning algorithms to solve the problem: the K-means algorithm, the logistic regression algorithm, the 
linear regression algorithm, and the decision tree algorithm. The first three algorithms will be used to find significant data 
trends, and the final algorithm will be used to figure out what decisions could be made to increase voter turnout for each respective
side (Democrat or Republican).

Thus far, our experimental results indicate that big headways could be made on both sides by two factors. The first is to increase 
those parties' respective ground involvement with voters of the opposite party, and the second factor is to increasingly address
the issues that are most important to a voter. As far as how to target with the ground game with regards to voter turnout, the 
results are inconclusive. While those with a lower voter turnout may be more receptive to a change in opinion, they are also less 
likely to actually go out and vote.

Many of the existing solutions involve enhancing engagement, and providing various forms of civic education. Civic education; 
especially, is useful in changing voters' minds. This is because in allowing voters to understand how the political systems work,
it alleviates their fears surrounding the opposite parties' policy positions, as well as ensuring they don't think they are giving
power to people they believe should not have it.

\section{Related Work}
% This section summarizes existing solutions to the problem or similar problems. Please try to categorize these existing techniques and provide some discussion on the pros and cons of them. Don't forget to include references to any existing work you mention.  
Solutions to this problem can be organized into two different categories: increasing engagement, and increasing civic education.

Solutions for increasing engagement include increasing communication, giving surveys, providing voting incentives, and maintaining
contact after the election. Frequent engagement doesn't just encourage voter turnout, but also provides voters with the opportunity
to engage with representatives of the other party, who may be able to answer their questions and allow them to see things from a 
different perspective. 

Solutions for increasing civic education include civic education campaigns, as well as clear explanations of the positions of
party candidates. This helps to ease voters' fears regarding the policies of various candidates and allows them to consider different
perspectives. Furthermore, increasing civic education allows voters to feel more secure in their vote as they will understand how the
checks and balances system works.

\section{Our Solution}
%This section elaborates your solution to the problem.

\subsection{Description of Dataset}
%This subsection describes the dataset that you will use. In addition to the source of the dataset, it is also expected to include discussions on data proprocessing, e.g.,  statistical properties or visualization of the raw data, observation of some issues with the data (e.g., missing features, irrelevant features, outliers, etc.), some feature engineering to fix the problems (e.g., filling missing data, encoding of strings or characters into numerical values, normalization, and other operations as appropriate).
The datasets used in this analysis were derived from voter turnout and population density data across various counties. The voter turnout data comes from the official primary election held on June 4, 2024, while the population density data was sourced from Census 2020 and 2022 estimates. 

During preprocessing, a few issues in the dataset were identified and corrected. Some records had missing values in the "Number of Children" field, which could have affected analyses related to family size and voting behavior. Additionally, there were extreme values in the "Salary" column that could have distorted our findings.

To address these issues, missing values in the "Number of Children" field were filled using the median, ensuring the data remained representative. Outliers in the "Salary" column were capped at a reasonable level to prevent skewed analysis.

Categorical data, such as gender, marital status, occupation, and party membership, were converted into numerical values for consistency. Continuous features like age and salary were normalized to ensure each feature contributed equally to the analysis. Data visualization techniques were used to check distributions, ensuring the data was ready for accurate analysis. These preprocessing steps cleaned and standardized the dataset, allowing for meaningful insights into voting patterns and demographics.



\subsection{Machine Learning Algorithms}
%This subsection describes machine learning algorithms that you plan to use. For each ML algorithm, briefly 1) explain why it might be appropriate for the problem and 2) describe your main design. For example, if it is neural network, provide the network structure and your initial choice of some key parameters (e.g., activation function to use, number of layers, number of hidden nodes of each layer). You may change the parameters during the training process.  

Here, we will implement four machine learning algorithms. We will implement K-mean clustering, logistic regression, linear regression and decision tree to gain insights of the data. K-means will be used to divide the data into clusters to look for some trends and patterns into urban and rural voters. K-means is an appropriate to look for patterns and insights on data. Either the Silhouette score or the Elbow method will be used to determine to number of clusters or both to compare the difference.
Logistic regression and decision tree will be used to predict if people in rural will turnout and also to be used to see if urban people will be predicted to be turnout or not and use that result to see if we can increase the turnout for both groups by 5\%. We will use gridsearchCV or similar to find the optimal hyperparameters for the algorithms. These classification algorithms are appropriate because, they will predict to see if the people will likely turnout or not and ultimately predict if we can increase turnout for both groups by 5\%. Linear regression can be use to to explore the actions that will increase turn out rates. This can be used to see if the turnout rate is increased as other parameters increases.

\subsubsection{Linear Regression}
We plan to employ linear regression to model the relationship between various demographic factors and voter turnout rate. This method is chosen because:

\begin{itemize}
    \item It will allow us to quantify the impact of multiple independent variables on voter turnout.
    \item The coefficients will provide interpretable insights into the importance of each factor.
    \item It can handle both continuous and categorical variables (after appropriate encoding).
\end{itemize}
For our analysis, we will focus on key numerical variables including age, salary, number of children, and voter turnout rate. We plan to normalize these features to ensure they are on the same scale. Categorical variables like gender, marital status, and occupation will be one-hot encoded to be usable in the linear regression model.

In our preliminary analysis, we observed that the age distribution differs between urban and rural areas, which may impact voting patterns. We are currently working on visualizations to illustrate these differences.


Our linear regression model will take the form:

\begin{equation}
\text{Voter Turnout Rate} = \beta_0 + \beta_1(\text{Age}) + \beta_2(\text{Salary}) + \beta_3(\text{NumChildren}) + ... + \epsilon
\end{equation}

Where $\beta_0$ is the intercept, $\beta_1, \beta_2,$ etc. are the coefficients for each feature, and $\epsilon$ is the error term.

\subsubsection{Logistic Regression}
We plan to use logistic regression to help us with two main tasks:

\begin{itemize}
    \item Predicting Voter Turnout: We’ll use it to figure out if a voter is likely to show up at the polls based on different demographic and engagement factors.
    \item Predicting Party Affiliation: It will also help us predict whether a voter is more likely to vote Republican or Democrat using the same set of factors.
\end{itemize}

We chose logistic regression for these tasks because:
\begin{itemize}
    \item It’s great for binary classification problems, which is perfect for our yes/no questions about voter turnout and party choice.
    \item The results are easy to understand, as the coefficients show how different factors influence the likelihood of turnout and party preference.
    \item It can work with both numerical data (like age and salary) and categorical data (like marital status and party membership) after we encode the categories properly.
\end{itemize}

For our analysis, we'll focus on important variables such as age, salary, number of children, marital status, occupation, party membership, and engagement with ground campaigns. We'll convert categorical variables into numerical values (one-hot encoding) and normalize continuous features like age and salary to make sure everything is on the same scale.




\subsection{Implementation Details}
%This subsection describes details of your implementation. Please focus on how you test and validate the performance, tune the hyperparameters, and select the best-performing models. Elaborate on techniques that you apply to improve the performance and explain why you use these techniques. You include few most important results/figures to illustrate your idea but do not let figures/tables dominate the content of the report. You can include few lines of critical code if needed. But please avoid paste lengthy code in your report. Please make sure the figures/tables/code snapshots are of appropriate size including the font size.
\subsubsection{Linear Regression Implementation}
We are in the process of implementing the linear regression model using scikit-learn in Python. Our planned process involves:

\begin{enumerate}
    \item Preprocessing the data, including handling missing values and encoding categorical variables.
    \item Splitting the data into training (80\%) and testing (20\%) sets.
    \item Fitting the model on the training data.
    \item Evaluating the model's performance on the test data using R-squared and Mean Squared Error (MSE) metrics.
\end{enumerate}

While we haven't yet obtained results, we anticipate that our analysis will reveal differences in the factors influencing voter turnout in urban versus rural areas. We expect variables such as age, education level, and income to have significant impacts, but the magnitude of these effects may vary between urban and rural settings.

In our next steps, we plan to:
\begin{itemize}
    \item Complete the data preprocessing steps
    \item Implement the linear regression model
    \item Analyze the results and create visualizations to illustrate our findings
    \item Compare the performance of linear regression with other machine learning models our team is implementing
\end{itemize}

So far, for fitting a basic k-means and using the Silhouette score, it found optimal k to be 3, so 3 different distinct groups. I'm thinking of also trying GMM alongside of the k-means because of the nature of k-means having a "hard" answer of having to belong in a group as oppose to GMM with a more probablistic approach.


\section{Comparison}  
This section includes the following: 1) comparing the performance of different machine learning algorithms that you used, and 2) comparing the performance of your algorithms with existing solutions if any. Please provide insights to reason about why this algorithm is better/worse than another one.

\section{Future Directions}
This section lays out some potential directions for further improving the performance. You can image what you may do if you were given extra 3-6 months.

\section{Conclusion}
This section summarizes this project, i.e., by the extensive experiments and analysis, do you think the problem is solved well? which algorithm(s) might be better suitable for this problem? Which technique(s) may help further improve the performance? \\

Last but not the least, don't forget to include references to any work you mentioned in the report.
  

\bibliographystyle{IEEEtran}
\bibliography{}


\end{document}


